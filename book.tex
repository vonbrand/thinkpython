% LaTeX source for ``Think Python: An Introduction to Software Design''
% Copyright (c)  2008  Allen B. Downey.

% Permission is granted to copy, distribute and/or modify this
% document under the terms of the GNU Free Documentation License,
% Version 1.1  or any later version published by the Free Software
% Foundation; with no Invariant Sections, no Front-Cover Texts,
% and no Back-Cover Texts.

% This distribution includes a file named fdl.tex that contains the text
% of the GNU Free Documentation License.  If it is missing, you can obtain
% it from www.gnu.org or by writing to the Free Software Foundation,
% Inc., 59 Temple Place - Suite 330, Boston, MA 02111-1307, USA.
%

%\documentclass[10pt,b5paper]{book}
\documentclass[10pt]{book}
\usepackage[width=5.5in,height=8.5in,
  hmarginratio=3:2,vmarginratio=1:1]{geometry}

\usepackage{pslatex}
\usepackage{url}
\usepackage{fancyhdr}
\usepackage{color}
\usepackage{graphicx}
\usepackage{amsmath, amsthm, amssymb}
\usepackage{exercise}
\usepackage{makeidx}
\usepackage{setspace}
\usepackage{hevea}
\usepackage{upquote}

\newcommand{\thetitle}{Think Python: An Introduction to Software Design}
\newcommand{\theversion}{1.1.14}
% -----------------------------
\newcommand{\FR}{\color{blue} \normalsize}
\newcommand{\EN}{\color{red} \normalsize}
\newcommand{\UN}{\color{black}\normalsize}
% -----------------------------
\makeindex

\begin{document}
\input{latexonly}
% ===============================================
\EN
\input{title.hxv}
% ===============================================
\EN
\chapter{Preface}
\input{preface_downey.hxv}
% ===============================================

% TABLE OF CONTENTS
\UN
\begin{latexonly}

\tableofcontents

\clearemptydoublepage

\end{latexonly}

% START THE BOOK
\mainmatter

% ===============================================
\EN
\chapter{The way of the program}
\input{TheWay.hxv}
% ===============================================
\EN
\chapter{Variables, expressions and statements}
\input{variables.hxv}
% ===============================================
\EN
\chapter{Functions}
\input{functions.hxv}
% ===============================================
\EN
\chapter{Case study: interface design}
\label{turtlechap}
\input{turtlechap.hxv}
% ===============================================
\EN
\chapter{Conditionals and recursion}
\input{recursion.hxv}
% ===============================================
\EN
\chapter{Fruitful functions}
\label{fruitchap}
\input{fruit.hxv}
% ===============================================
\EN
\chapter{Iteration}
\index{iteration}
\input{iteration.hxv}
% ===============================================
\EN
\chapter{Strings}
\label{strings}
\input{strings.hxv}
% ===============================================
\EN
\chapter{Case study: word play}
\input{wordlist.hxv}
% ===============================================
\EN
\chapter{Lists}
\input{lists.hxv}
% ===============================================
\EN
\chapter{Dictionaries}
\input{dictionary.hxv}
% ===============================================
\EN
\chapter{Tuples}
\label{tuplechap}
\input{tuple.hxv}
% ===============================================
\EN
\chapter{Case study: data structure selection}
\input{casedata.hxv}
% ===============================================
\EN
\chapter{Files}
\input{files.hxv}
% ===============================================
\EN
\chapter{Classes and objects}
\input{classobj.hxv}
% ===============================================
\EN
\chapter{Classes and functions}
\label{time}
\input{classfunc.hxv}
% ===============================================
\EN
\chapter{Classes and methods}
\input{classmethod.hxv}
% ===============================================
\EN
\chapter{Inheritance}
\input{inheritance.hxv}
% ===============================================
\EN
\chapter{Case study: Tkinter}
\input{tkinter.hxv}
% ===============================================
\UN
\appendix
% ===============================================
\EN
\chapter{Debugging}
\input{debug.hxv}
% ===============================================
\UN
\printindex

\clearemptydoublepage
%\blankpage
%\blankpage
%\blankpage


\end{document}
